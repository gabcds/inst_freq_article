%% LaTeX.tex
%% Example for the proceedings of the 19th Brazilian Congress of Thermal Sciences and Engineering
%% ENCIT 20222
%% November, 6-10, 2022 (Online)
%% Based on the template of the proceedings of ENCIT2020

\documentclass[10pt,fleqn,a4paper,twoside]{article}
\usepackage{abcm}
\def\shortauthor{F. Author, S. Author and T. Author (update this heading accordingly)}
\def\shorttitle{Paper Short Title (First Letters Uppercase, make sure it fits in one line)}

\begin{document}
\fphead
\hspace*{-2.5mm}\begin{tabular}{||p{\textwidth}}
\begin{center}
\vspace{-4mm}
\title{Enhancing Experimental and Numerical Data Validation through Acoustic
Noise Signal Demodulation for Estimating Drone Propeller Rotational Speed} %(XXXX is the manuscript number. It will be available after the extended abstract submission and must be placed for the final paper submission.)
\end{center}
\authors{Gabriel Costa da Silva} \\
\authors{Mateus Grassano Lattari} \\
\institution{Federal Univerisity of Santa Catarina, UFSC} \\
\institution{gabriel.silva@polo.ufsc.br} \\
\institution{mateus.grassano@polo.ufsc.br} \\
\\
\authors{Lucas Bonomo Araújo} \\
\authors{Julio Codiolo} \\ 
\authors{Racquel Knust Domingues} \\
\authors{Augusto Barth Beck} \\ 
\institution{Federal Univerisity of Santa Catarina, UFSC} \\ %(If all authors are from the same institution, the "Institution and address" must be placed only once.)
\institution{lucas.bonomo@lva.ufsc.br} \\
\institution{julio.cordioli@ufsc.br} \\
\institution{racquel.knust@lva.ufsc.br} \\
\institution{augusto.barth.beck@gmail.com} \\
\\
\abstract{\textbf{Abstract.} This study addresses a novel approach for estimating the rotational speed of
small-scale propellers, typically found in drones, through the analysis of their
acoustic noise signal. Accurately determining the instantaneous rotational speed
of propellers in anechoic wind-tunnels poses significant challenges due to
inherent experimental rotational speed fluctuations. These fluctuations can
distort harmonic peaks levels, or deteriorate the process of applying comparable
techniques when validating constant rotational speed numerical simulations with
experimental data. This can be overcome by the knowledge of the propeller
instantaneous rotational speed, allowing the signal to be resampled, correcting
it to a constant rotational speed. Measured tachometer data is often not
available nor reliable, as the use of external devices is not always feasible
due to space constraints, costs, and sensitivity to adverse environmental
conditions. An alternative approach is to directly estimate the propeller
rotational speed from the measured acoustic signal, which is the focus of this
study. The proposed methodology is based on the signal demodulation, which is a
tacholess method that calculates the Hilbert Transform of the acoustic signal to
obtain the frequency and phase related to the shaft rotation. To evaluate the
technique, a synthetic propeller noise data are generated with a previous
established rotational speed fluctuation, allowing a characteristic error for the
algorithm predicted instantaneous rotation to be obtained. Secondly, the process
is repeated for a real propeller noise signal, and the results are compared with
the actual rotational speed measurement obtained with the tachometer. Finally,
the obtained instantaneous rotation is employed to resample the experimental
signal, allowing it to be suitable for validating numerical simulated signals.
The spectra obtained from both signals are then compared, and the signal
components are evaluated using a Time Synchronous Averaging (TSA) analysis.
Preliminary results indicate the consistency and feasibility of the technique.}\\
\\
\keywords{\textbf{Keywords:} Propeller noise, frequency estimation,signal processing, aerodynamic noise.}\\
\end{tabular}

\section{INTRODUCTION}
In the ever-evolving realm of drone technology, the precise estimation of propeller rotational speed stands as a pivotal challenge. The need of knowing this information provides further analysis, such as dynamic control, precise noise sources identifications with decomposition techniques and failure prediction.

Upon scrutinizing the acoustic traits of noise produced by fully electric propulsion systems, it becomes apparent that the primary sources are the interactions between the blades and the airflow, encompassing turbulence and vortical effects. The dominant aspect of the noise spectrum comprises the tonal rendition, characterized by multiples of the blade-pass frequency (BPF), signifying a periodic signal. In contrast, the broadband feature, originating from the blade interactions, disperses energy throughout all frequency bands, exhibiting inherent stochasticity. In order to better analyze the noise sources, the features must be separated, which can express plenty dificulties, upon rotational speed fluctuations.

In this context, the Time Synchronous Averaging Method (TSA) \citep{MCFADDEN1987173} finds extensive application in rotors operating at constant rotational speeds, owing to its straightforward implementation and effectiveness in isolating peaks. The method operates by averaging segments of acoustic data corresponding to a single rotation length in the time domain. However, its efficacy diminishes when applied to systems with varying speeds, as the irregular periodicity of segments undermines its performance. \cite{SHARMA2016560} proposed a tonal and broadband components TSA-based decomposition in order to calculate fault indicators in gears, which considerates the rotational frequency fluctuation, therefore, this technique takes in account the tachometer signal, using n pulses per revolution. With this device it is possible to track the angular position of any shaft. 

Small-scale propellers, typically found in drones, necessitate accurate measurement techniques amidst the backdrop of inherent experimental fluctuations. Traditional methods, reliant on tachometer data, often falter due to practical constraints and environmental sensitivities, prompting a quest for alternative methodologies. \cite{article} presents a complete analysis on various methods that discart the need of a tacho signal.

\cite{Urbanek2011ComparisonOA} investigate three major instantaneous frequency estimation techiniques without any phase markers use in wind turbines speed tracking application. The spectrogram-based method proceeds with a maxima tracking due to the fact the peaks with the highest energy on the spectrogram should correspond to the value of the instantaneous frequency at each moment in time.

In another approach,\cite{BONNARDOT2005766} proposed a tacholess technique of estimating the instantaneous rotation of a shaft with limited frequency fluctuations. The method is based on the phase demodulation and utilizes the Hilbert Transform, a matematical instrument to obtain the imaginary part of the analytic signal, which corresponds to the phase of the signal. The main achievement of this technique is that extincts the need of a tachometer, the instant rotation can calculated based on the shaft vibration data. This work analyses the merits and the feasibility of this method.

This work is organized as follows: Section 2 presents the methodology of the phase demodulation technique, describing in details the physics behind the numerical calculation. Section 3 describes some setups of both numerical and experimental data and Section 4 compares the results with the tachometer real information, as well as evaluates the components of the resampled signal with Time Synchronous Averaging (TSA) techiniques. Finally, Section 5 concludes the consistency of the demodulation method.

\section{METHODOLOGY}
The primary method used for determining the instantaneous frequencies of a two-bladed propeller is the Demodulation Method \citep{BONNARDOT2005766}, a robust and effective procedure that allows for a detailed and precise analysis of dynamic variations over time.

The techinique consists in 5 steps:

\subsection{Band Inspection and Signal Filtering}

In this step, a characteristic harmonic of the signal is selected, with this study focusing on the first Blade Pass Frequency (BPF), which represents the harmonic with the highest energy. Subsequently, through a detailed visual analysis of the spectrum, a band-pass filter is applied to precisely isolate the region corresponding to the shaft rotation harmonic, along with its controlled frequency variations. This approach ensures a focused and accurate examination of the relevant dynamic behaviors.

It is important to add that the band of interest choice is primarily empirical, which results in testing band percentages and assess the performance of the ones who fit the problem the most. In this study we manage to compare multiple band percentages with the experimental instantaneous rotational frequency of the propeller.

\subsection{Analytic Signal Estimation}


\bibliographystyle{abcm}
\renewcommand{\refname}{}
\bibliography{bibfile}

\section{RESPONSIBILITY NOTICE}

The following text, properly adapted to the number of authors, must be included in the last section of the paper:

The author(s) is (are) solely responsible for the printed material included in this paper.

\end{document}
