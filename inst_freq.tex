%% LaTeX.tex
%% Example for the proceedings of the 19th Brazilian Congress of Thermal Sciences and Engineering
%% ENCIT 20222
%% November, 6-10, 2022 (Online)
%% Based on the template of the proceedings of ENCIT2020

\documentclass[10pt,fleqn,a4paper,twoside]{article}
\usepackage{abcm}
\def\shortauthor{F. Author, S. Author and T. Author (update this heading accordingly)}
\def\shorttitle{Paper Short Title (First Letters Uppercase, make sure it fits in one line)}

\begin{document}
\fphead
\hspace*{-2.5mm}\begin{tabular}{||p{\textwidth}}
\begin{center}
\vspace{-4mm}
\title{Enhancing Experimental and Numerical Data Validation through Acoustic
Noise Signal Demodulation for Estimating Drone Propeller Rotational Speed} %(XXXX is the manuscript number. It will be available after the extended abstract submission and must be placed for the final paper submission.)
\end{center}
\authors{Gabriel Costa da Silva} \\
\authors{Mateus Grassano Lattari} \\
\institution{Federal Univerisity of Santa Catarina, UFSC} \\
\institution{gabriel.silva@polo.ufsc.br} \\
\institution{mateus.grassano@polo.ufsc.br} \\
\\
\authors{Lucas Bonomo Araújo} \\
\authors{Julio Codiolo} \\ 
\authors{Racquel Knust Domingues} \\
\authors{Augusto Barth Beck} \\ 
\institution{Federal Univerisity of Santa Catarina, UFSC} \\ %(If all authors are from the same institution, the "Institution and address" must be placed only once.)
\institution{lucas.bonomo@lva.ufsc.br} \\
\institution{julio.cordioli@ufsc.br} \\
\institution{racquel.knust@lva.ufsc.br} \\
\institution{augusto.barth.beck@gmail.com} \\
\\
\abstract{\textbf{Abstract.} This study addresses a novel approach for estimating the rotational speed of
small-scale propellers, typically found in drones, through the analysis of their
acoustic noise signal. Accurately determining the instantaneous rotational speed
of propellers in anechoic wind-tunnels poses significant challenges due to
inherent experimental rotational speed fluctuations. These fluctuations can
distort harmonic peaks levels, or deteriorate the process of applying comparable
techniques when validating constant rotational speed numerical simulations with
experimental data. This can be overcome by the knowledge of the propeller
instantaneous rotational speed, allowing the signal to be resampled, correcting
it to a constant rotational speed. Measured tachometer data is often not
available nor reliable, as the use of external devices is not always feasible
due to space constraints, costs, and sensitivity to adverse environmental
conditions. An alternative approach is to directly estimate the propeller
rotational speed from the measured acoustic signal, which is the focus of this
study. The proposed methodology is based on the signal demodulation, which is a
tacholess method that calculates the Hilbert Transform of the acoustic signal to
obtain the frequency and phase related to the shaft rotation. To evaluate the
technique, a synthetic propeller noise data are generated with a previous
established rotational speed fluctuation, allowing a characteristic error for the
algorithm predicted instantaneous rotation to be obtained. Secondly, the process
is repeated for a real propeller noise signal, and the results are compared with
the actual rotational speed measurement obtained with the tachometer. Finally,
the obtained instantaneous rotation is employed to resample the experimental
signal, allowing it to be suitable for validating numerical simulated signals.
The spectra obtained from both signals are then compared, and the signal
components are evaluated using a Time Synchronous Averaging (TSA) analysis.
Preliminary results indicate the consistency and feasibility of the technique.}\\
\\
\keywords{\textbf{Keywords:} Propeller noise, frequency estimation,signal processing, aerodynamic noise.}\\
\end{tabular}

\section{INTRODUCTION}
In the ever-evolving realm of drone technology, the precise estimation of propeller rotational speed stands as a pivotal challenge. The need of knowing this information provides further analysis, such as dynamic control, precise noise sources identifications with decomposition techniques and failure prediction.

Upon scrutinizing the acoustic traits of noise produced by fully electric propulsion systems, it becomes apparent that the primary sources are the interactions between the blades and the airflow, encompassing turbulence and vortical effects. The dominant aspect of the noise spectrum comprises the tonal rendition, characterized by multiples of the blade-pass frequency (BPF), signifying a periodic signal. In contrast, the broadband feature, originating from the blade interactions, disperses energy throughout all frequency bands, exhibiting inherent stochasticity. In order to better analyze the noise sources, the features must be separated, which can express plenty dificulties, upon rotational speed fluctuations.

In this context, the Time Synchronous Averaging Method (TSA) \citep{MCFADDEN1987173} finds extensive application in rotors operating at constant rotational speeds, owing to its straightforward implementation and effectiveness in isolating peaks. The method operates by averaging segments of acoustic data corresponding to a single rotation length in the time domain. However, its efficacy diminishes when applied to systems with varying speeds, as the irregular periodicity of segments undermines its performance. \cite{SHARMA2016560} proposed a tonal and broadband components TSA-based decomposition in order to calculate fault indicators in gears, which considerates the rotational frequency fluctuation, therefore, this technique takes in account the tachometer signal, using n pulses per revolution. With this device it is possible to track the angular position of any shaft. 

Small-scale propellers, typically found in drones, necessitate accurate measurement techniques amidst the backdrop of inherent experimental fluctuations. Traditional methods, reliant on tachometer data, often falter due to practical constraints and environmental sensitivities, prompting a quest for alternative methodologies. \cite{article} presents a complete analysis on various methods that discart the need of a tacho signal.

\cite{Urbanek2011ComparisonOA} investigate three major instantaneous frequency estimation techiniques without any phase markers use in wind turbines speed tracking application. The spectrogram-based method proceeds with a maxima tracking due to the fact the peaks with the highest energy on the spectrogram should correspond to the value of the instantaneous frequency at each moment in time.

In another approach,\cite{BONNARDOT2005766} proposed a tacholess technique of estimating the instantaneous rotation of a shaft with limited frequency fluctuations. The method is based on the phase demodulation and utilizes the Hilbert Transform, a matematical instrument to obtain the imaginary part of the analytic signal, which corresponds to the phase of the signal. The main achievement of this technique is that extincts the need of a tachometer, the instant rotation can calculated based on the shaft vibration data. This work analyses the merits and the feasibility of this method.

This work is organized as follows: Section 2 presents the methodology of the phase demodulation technique, describing in details the physics behind the numerical calculation. Section 3 describes some setups of both numerical and experimental data and Section 4 compares the results with the tachometer real information, as well as evaluates the components of the resampled signal with Time Synchronous Averaging (TSA) techiniques. Finally, Section 5 concludes the consistency of the demodulation method.

\section{TEXT FORMAT}

The manuscripts should be written in English, typed in A4 size pages, using font Times New Roman, size 10, except for the title, authors affiliation, abstract and keywords, for which particular formatting instructions are indicated above. Single space between lines is to be used throughout the text.

The text block that contains the title, the authors' names and affiliation, the abstract and the keywords must be indented 0.1 cm from the left margin and marked by a leftmost black line border of width 2 1/4 pt.

The first page must have a top margin of 3 cm and all the other margins (left, right and bottom) must have 2 cm. All the other pages must be set with all margins equal to 2 cm.

PAGES {\bf SHOULD NOT} BE NUMBERED

The body of the text must be justified. The first line of each paragraph must be indented by 0.5 cm. Sufficient information must be provided directly in the text, or by reference to widely available published work. Footnotes should be avoided.

All the symbols and notation must be defined in the text. Physical quantities must be expressed in the SI (metric) units. Mathematical symbols appearing in the text must be typed in italic style.

Bibliographic references should be cited in the text by giving the last name of the author(s) and the year of publication, according to the following examples: ``In a recent work~\citep{BandarraFilho2011}\dots'' or ``Recently, \citet{BandarraFilho2011}\dots''. In the case of three or more authors, the form ``\citep{CavaliniJunior2015}'' should be used. Two or more references having the same authors and publication year must be distinguished by appending ``a'', ``b'', etc., to the year of publication. For exemple: ``In papers ~\citep{Santos2013a} and \citep{Santos2013b}\dots''.

Acceptable references include journal articles ~\citep{MLA04}, numbered papers, dissertations and theses~\citep{CavaliniJunior2013,coelho2017}, published conference proceedings, preprints from conferences, books~\citep{McConnell.Varoto.2008} and submitted articles (if the journal is identified).

References should be listed at the end of the paper according to instructions provided in Section 4.

\subsection{Section titles and subtitles}

The section titles and subtitles must be aligned at left, typed with Times New Roman, size 10, bold style font. They must be numbered using Arabic numerals separated by points. No more than 3 sublevels should be used. One single line must be included above and bellow each section title/subtitle.

\subsection{Mathematical equations}

The mathematical equations must be indented by 0.5 cm from the left margin. They must be typed using Times New Roman, italic, size 10 pt. font.\ Arabic numerals must be used as equation numbers, enclosed between parentheses, right-aligned, as shown in the examples below. Equations should be referred to either as ``Eq.~(\ref{eq1})'' in the middle of a phrase or as ``Equation~(\ref{eq1})'' in the beginning of a sentence. Matrix and vector quantities can be indicated either by brackets and braces, as in Eq.~(\ref{eq1}), or in bold style, as in Eq.~(\ref{eq2}). Symbols used in the equations must be defined immediately before or after their first appearance.

One single line must be included above and below each equation.
\begin{equation}
[M]\{\ddot{x}\}+[C]\{\dot{x}(t)\}+[K]\{x(t)\}={f(t)} 
\label{eq1}
\end{equation}
\begin{equation}
\mathbf{M\ddot{x}}(t)+\mathbf{C\dot{x}}(t)+\mathbf{Kx}(t)=\mathbf{f}(t) 
\label{eq2}
\end{equation}

\subsection{Figures and tables}

Figures and tables should be placed in the text as close as possible to the point they are first mentioned and must be numbered consecutively in arabic numerals. Figures must be referred to either as ``Fig.~\ref{fig1}'' in the middle of a phrase or as ``Figure~\ref{fig1}'' in the beginning of a sentence. The figures themselves as well as their captions must be centered in the breadth-wise direction. The captions of the figures should not be longer than 3 lines.

The legend for the data symbols as well as the labels for each curve should be included into the figure. Lettering should be large enough for ease reading. All units must be expressed in the S.I. (metric) system.

One blank line must be left before and after each figure.
\begin{figure}[h!]
\centering
\includegraphics[angle=0, scale=0.320]{Figures/figure.jpeg}
\caption{United States crude oil imports from Norway versus number of drivers killed in collision with railway train. Available from: http://tylervigen.com/spurious-correlations}
\label{fig1}
\end{figure}

Color figures and high-quality photographs can be included in the paper. To reduce the file size and preserve the graphic resolution, figures must be saved into GIF (figures with less than 16 colors) or JPEG (for higher color density) files before being inserted in the manuscript.

Tables must be referred to either as ``Tab.~\ref{tab1}'' in the middle of a phrase or as ``Table~\ref{tab1}'' in the beginning of a sentence.  The tables themselves as well as their titles must be centered in the breadth-wise direction. The titles of the tables should not be longer than 3 lines. The font style and size used in the tables must be similar (both in size and style) to those used in the text body. Units must be expressed in the S.I.\ (metric) system. Explanations, if any, should be given at the foot of the tables, not within the tables themselves.

One blank line must be left before and after each table.

The style of table borders is left free. An example is given in Tab.~\ref{tab1}.
\begin{table}[!h]
\centering
\caption{Experimental results for flexural properties of CFRC-4HS and CFRC-TWILL composites. \protect\\Span/depth ratio = 35:1. Average results of 7 specimens.}
\begin{tabular}{|c|c|c|}
\hline
Composite Properties & CFRC-TWILL & CFRC-4HS\\
\hline
Flexural Strength (MPa)$^{(1)}$ & 209$\pm$ 10 & 180 $\pm$  15\\
\hline
Flexural Modulus (GPa)$^{(1)}$ & 57.0 $\pm$ 2.8 & 18.0 $\pm$  1.3\\
\hline
Mid-span deflection at the failure stress (mm) & 2.15 $\pm$  1.90 & 6.40 $\pm$  0.25\\
\hline
\end{tabular}
\\
\begin{tabular}{p{11cm}ll}
$^{(1)}$ measured at 25$^{o}$C & &
\end{tabular}
\label{tab1}
\end{table}

\section{ACKNOWLEDGEMENTS}
This optional section must be placed before the list of references.

\section{REFERENCES} 

The list of references must be introduced as a new section, located at the end of the paper. The first line of each reference must be aligned at left.  All the other lines must be indented by 0.5 cm from the left margin. All references included in the reference list must have been mentioned in the text.

References must be listed in alphabetical order, according to the last name of the first author. See the following examples:

\bibliographystyle{abcm}
\renewcommand{\refname}{}
\bibliography{bibfile}

\section{RESPONSIBILITY NOTICE}

The following text, properly adapted to the number of authors, must be included in the last section of the paper:

The author(s) is (are) solely responsible for the printed material included in this paper.

\end{document}
